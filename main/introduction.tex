%% LaTeX2e class for student theses
%% sections/introduction.tex
%%!TEX root = ./thesis.tex
%% Based on SDQ KIT Template by Dr.-Ing. Erik Burger (Version 1.4, 2023-06-19)
%%
%% Karlsruhe Institute of Technology
%% Institute for Automation and Applied Informatics (IAI)
%% IT Methods and Components for Energy Systems (IT4ES) Research Group
%%
%% Gökhan Demirel
%% goekhan.demirel@kit.edu
%%
%% Adaptation Version 1.1, 19.02.2024

\chapter{Introduction}
\label{ch:Introduction}

.

\section{Motivation and Objectives}
To separate parts of text in \gls{latex} \gls{LV} \gls{example}, please use two line breaks in your source code.
They will then be set with correct indentation.
Do \emph{not} use:
\begin{itemize}
  \itemsep0em
  \item \texttt{\textbackslash\textbackslash}
  \item \texttt{\textbackslash parskip}
  \item \texttt{\textbackslash vskip}
\end{itemize} 
or other commands to manually insert spaces, since they break the layout of this template.

\section{Related Work}
\label{sec:Introduction:Related Work}
This template is based on \texttt{biblatex} and \texttt{biber}, which is preferred over the
outdated Bib\TeX{} software.

A citation: \cite{becker2008a} 

\section{Example: Figures}
\label{sec:Introduction:Figures}
\begin{figure}
\centering
\includegraphics[width=4cm]{logos/kitlogo_en_cmyk.pdf}
\caption{KIT logo}
\label{fig:kit0logo}
\end{figure}

A reference: The KIT logo is displayed in \autoref{fig:kit0logo}. 
(Use \code{\textbackslash autoref\{\}} for easy referencing.) 

\section{Example: Tables}
The \texttt{booktabs} package offers nicely typeset tables, as in \autoref{tab:atable}.

\label{sec:Introduction:Tables}
\begin{table}
\centering
\begin{tabular}{r l}
\toprule
abc & def\\
ghi & jkl\\
\midrule
123 & 456\\
789 & 0AB\\
\bottomrule
\end{tabular}
\caption{A table}
\label{tab:atable}
\end{table}

\section{Example: Formula}
One of the nice things about the Linux Libertine font is that it comes with
a math mode package.
\begin{displaymath}
f(x)=\Omega(g(x))\ (x\rightarrow\infty)\;\Leftrightarrow\;
\limsup_{x \to \infty} \left|\frac{f(x)}{g(x)}\right|> 0
\end{displaymath}

%% --------------------
%% | /Example content |
%% --------------------